\documentclass[notitlepage,aps,prd,nofootinbib]{revtex4-1}

\usepackage{subfig}
%\usepackage[colorinlistoftodos]{todonotes}
\usepackage{float}

%\usepackage[protrusion=true,expansion=true]{microtype}
\usepackage{amsmath}
\usepackage{amssymb}
\usepackage{bbm}
\usepackage{ulem}
%\usepackage{feynmp-auto}
%\usepackage{slashed}
%\usepackage[absolute,overlay]{textpos}
\usepackage[usenames, dvipsnames]{color}
\usepackage{graphicx}
\usepackage{listings}
\usepackage{epsfig}
\usepackage{hyperref}
%\usepackage{tikz}
\usepackage{enumerate}
%\usepackage{fixltx2e} % buggy
\usepackage[compatibility=false]{caption}
%\usepackage{subcaption} % doesn't work with subfigure
\usepackage{pdfpages}
%\usepackage{setspace}
\usepackage{verbatim}

% Turn off meaningless float warnings
\usepackage{silence}
\WarningFilter{revtex4-1}{Repair the float}

\DeclareRobustCommand{\orderof}{\ensuremath{\mathcal{O}}}

\definecolor{dukeblue}{RGB}{0,0,156}
\definecolor{dukedarkblue}{RGB}{0,26,87}
\definecolor{dukeblack}{RGB}{79,79,79}
\definecolor{dukegray}{RGB}{79,79,79}
\definecolor{dukesecbrown}{RGB}{217,200,158}
\definecolor{dukesecblue}{RGB}{127,169,174}

%\renewcommand*{\thefootnote}{\fnsymbol{footnote}}

%%%%%%%%%%%%%%%%%%%%%%%%%%%%%%%%%%%%%%%%%%%%%%%%%%%%%%%%%%%%%%%%%%%%%%%%%%%%%%%%%%%%%
\hypersetup{
    breaklinks,
    baseurl       = http://,
    pdfborder     = 0 0 0,
    pdfpagemode   = UseNone,% do not show thumbnails or bookmarks on opening
    pdfstartpage  = 1,
    bookmarksopen = true,
    bookmarksdepth= 2,% to show sections and subsections
% revtex needs author and title declared after \begin{document}, so have to hard code them...
%    pdfauthor     = {\@author},
%    pdftitle      = {\@title},
    pdfauthor     = {Matthew Epland},
    pdftitle      = {Phys 566 Final},
    pdfsubject    = {},
    pdfkeywords   = {}}


% Code import settings
%%%%%%%%%%%%%%%%%%%%%%%%%%%%%%%%%%%%%%%%%%%%%%%%%%%%%%%%%%%%%%%%%%%%%%%%%%%%%%%%%%%%%
\definecolor{mygreen}{rgb}{0,0.6,0}
\definecolor{mygray}{rgb}{0.5,0.5,0.5}
\definecolor{mymauve}{rgb}{0.58,0,0.82}

%\lstset{ %
\lstdefinestyle{python}{ %
  backgroundcolor=\color{white},   % choose the background color; you must add \usepackage{color} or \usepackage{xcolor}
  basicstyle=\scriptsize,          % the size of the fonts that are used for the code
  breakatwhitespace=false,         % sets if automatic breaks should only happen at whitespace
  breaklines=true,                 % sets automatic line breaking
  captionpos=b,                    % sets the caption-position to bottom
  commentstyle=\color{mygreen},    % comment style
  deletekeywords={...},            % if you want to delete keywords from the given language
  escapeinside={\%*}{*)},          % if you want to add LaTeX within your code
  extendedchars=true,              % lets you use non-ASCII characters; for 8-bits encodings only, does not work with UTF-8
  frame=single,	                   % adds a frame around the code
  keepspaces=true,                 % keeps spaces in text, useful for keeping indentation of code (possibly needs columns=flexible)
  keywordstyle=\color{blue},       % keyword style
  language=Python,                 % the language of the code
  otherkeywords={*,...},           % if you want to add more keywords to the set
  numbers=left,                    % where to put the line-numbers; possible values are (none, left, right)
  numbersep=5pt,                   % how far the line-numbers are from the code
  numberstyle=\tiny\color{mygray}, % the style that is used for the line-numbers
  rulecolor=\color{black},         % if not set, the frame-color may be changed on line-breaks within not-black text (e.g. comments (green here))
  showspaces=false,                % show spaces everywhere adding particular underscores; it overrides 'showstringspaces'
  showstringspaces=false,          % underline spaces within strings only
  showtabs=false,                  % show tabs within strings adding particular underscores
  stepnumber=5,                    % the step between two line-numbers. If it's 1, each line will be numbered
  stringstyle=\color{mymauve},     % string literal style
  tabsize=2,	                   % sets default tabsize to 2 spaces
%  title=\lstname                   % show the filename of files included with \lstinputlisting; also try caption instead of title
  title={\protect\filename@parse{\lstname}\protect\filename@base.\filename@ext},
  firstnumber=0,
%  linewidth=0.95\textwidth
  xleftmargin=0.01\textwidth,
  xrightmargin=0.01\textwidth
}

\lstdefinestyle{output}{ %
  backgroundcolor=\color{white},   % choose the background color; you must add \usepackage{color} or \usepackage{xcolor}
  basicstyle=\scriptsize,          % the size of the fonts that are used for the code
  breakatwhitespace=false,         % sets if automatic breaks should only happen at whitespace
  breaklines=true,                 % sets automatic line breaking
  captionpos=b,                    % sets the caption-position to bottom
  escapeinside={\%*}{*)},          % if you want to add LaTeX within your code
  frame=single,	                   % adds a frame around the code
  keepspaces=true,                 % keeps spaces in text, useful for keeping indentation of code (possibly needs columns=flexible)
  numbers=left,                    % where to put the line-numbers; possible values are (none, left, right)
  numbersep=5pt,                   % how far the line-numbers are from the code
  numberstyle=\tiny\color{mygray}, % the style that is used for the line-numbers
  rulecolor=\color{black},         % if not set, the frame-color may be changed on line-breaks within not-black text (e.g. comments (green here))
  stepnumber=5,                    % the step between two line-numbers. If it's 1, each line will be numbered
  tabsize=2,	                   % sets default tabsize to 2 spaces
%  title=\lstname                   % show the filename of files included with \lstinputlisting; also try caption instead of title
  title={\protect\filename@parse{\lstname}\protect\filename@base.\filename@ext},
  firstnumber=0,
%  linewidth=0.95\textwidth
  xleftmargin=0.01\textwidth,
  xrightmargin=0.01\textwidth
}

\newcommand{\degree}{\ensuremath{^{\circ}}}

% Select between raw and saved plots here
%\graphicspath{{../output/}}

%%%%%%%%%%%%%%%%%%%%%%%%%%%%%%%%%%%%%%%%%%%%%%%%%%%%%%%%%%%%%%%%%%%%%%%%%%%%%%%%%%%%%
\begin{document}

\title{PHYS 566 Final}
\author{Matthew Epland}
\affiliation{Department of Physics, Duke University, Durham, NC 27707, USA}

\date{\today}

\begin{abstract}
TODO
\end{abstract}\maketitle

\section{Introduction and Theory}
\label{sec:theory}
\subsection{Ising Model}
\label{subsec:ising}
The Ising model is the standard statistical mechanics first order model for a lattice of spins, such as the spins of atoms in a ferromagnet. It is unique in being both analytically solvable, for some cases\footnote{1D with periodic boundary conditions is fairly straight forward, however no 3D analytical solutions have yet been developed.}, and in having a phase transition. In the Ising model each lattice point can either be spin up, $s=+1$ or spin down, $s=-1$, and the Hamiltonian\footnote{Without the presence of an external field} (\ref{eq:E}) only considers nearest neighbor (NN) interactions with exchange energy $J$.

\begin{equation}
\label{eq:E}
E = -J \sum_{\mathrm{NN}} s_{i} s_{j}
\end{equation}

Due to the binary spin orientations, the magnetization of an Ising model lattice can easily be calculated as a simple average over the $N$ lattice points (\ref{eq:M}).

\begin{equation}
\label{eq:M}
M = \frac{1}{N} \sum_{j=1}^{N} s_{i} = N \langle s \rangle
\end{equation}

The above equations apply to specific lattices or microstates. As is common in statistical mechanics we can take averages over microstates $\alpha$ to produce more accurate thermal quantities, in particular $\langle E \rangle$ (\ref{eq:ave_E}) and $\langle E^{2} \rangle$ (\ref{eq:ave_E2}).

\begin{align}
\langle E \rangle &= \frac{1}{N_{\mathrm{microstates}}} \sum_{\alpha} E_{\alpha} \label{eq:ave_E} \\
\langle E^{2} \rangle &= \frac{1}{N_{\mathrm{microstates}}} \sum_{\alpha} E_{\alpha}^{2} \label{eq:ave_E2}
\end{align}

We are interested in $\langle E \rangle$ and $\langle E^{2} \rangle$ because they can be used to compute the squared variance of $E$, $\sigma_{E}^{2}$ (\ref{eq:sigmaE2}), which can itself be used to compute the specific heat $C$ at temperature $T$ via the fluctuation--dissipation theorem\footnote{$k_{B}$ is the standard Boltzmann constant.} (\ref{eq:C}).

\begin{align}
\sigma_{E}^{2} &=  \langle E^{2} \rangle - \langle E \rangle^{2} \label{eq:sigmaE2} \\
C &= \frac{\sigma_{E}^{2}}{k_{B} T^{2}} \label{eq:C}
\end{align}

The Ising model can exhibit a phase transition between $M=\langle s \rangle \sim 0$ disordered spins and $\left|M\right|\sim N,\,\left|\langle s \rangle\right| \sim 1$ ordered spins, ie spontaneous magnetization, as $T$ changes through some critical $T_{C}$. It is possible to analytically derive $T_{C}$, and indeed this is one of the Ising models main positives, however we will not be discussing analytical approaches in this work.\footnote{References can easily be found in the literature, one example is Pathria and Beale.} Instead we will be focused on simulating the Ising model numerically with the Metropolis algorithm.

\subsection{Metropolis Algorithm}
\label{subsec:met_alg}
TODO\footnote{The Von Neumann neighborhood consists of the four normal lattice points, to the left, right, up, and down of the original point. The Moore neighborhood adds the four diagonal neighbors for a total of eight.}

$N=n^{2}$

$\Delta E = E_{\mathrm{flipped}} - E_{\mathrm{original}}$


\begin{equation}
\label{eq:p}
p = \exp\left(\frac{-\Delta E}{k_{B} T}\right)
\end{equation}



\subsubsection{Convergence Criteria}
\label{subsubsec:convergence}
TODO


\begin{equation}
\label{eq:histE}
\langle \Delta E \rangle_{\mathrm{sweeps}} = \frac{1}{N_{\mathrm{sweeps}}} \sum_{i=0}^{N_{\mathrm{sweeps}}-1} \left|\frac{ E_{i+1} - E_{i}}{E_{i+1}}\right|
\end{equation}



\section{Results}
\label{sec:results}
TODO


\subsection{Part A}
\label{subsec:results_part_a}
TODO


\begin{figure}[!htbc]
  \centering
  \includegraphics[width=.72\textwidth]{../output/plots_for_paper_von_neumann/part_a/initial.pdf}
	{\par\nobreak\rule[9pt]{35em}{0.5pt}\vspace{-5mm}}
	\caption{Initial lattice.}
	\label{fig:initial}
\end{figure}

\begin{figure}[!htbc]
  \centering
  \includegraphics[width=.72\textwidth]{{../output/plots_for_paper_von_neumann/part_a/converged_T6.00}.pdf}
	{\par\nobreak\rule[9pt]{35em}{0.5pt}\vspace{-5mm}}
	\caption{Lattice at $T=6.0$. While the convergence condition was not met before the simulation timed out the lattice seems reasonable.}
	\label{fig:T6.0}
\end{figure}

\begin{figure}[!htbc]
  \centering
  \includegraphics[width=.72\textwidth]{{../output/plots_for_paper_von_neumann/part_a/converged_T3.60}.pdf}
	{\par\nobreak\rule[9pt]{35em}{0.5pt}\vspace{-5mm}}
	\caption{Lattice at $T=3.6$. Here the simulation met the convergence condition.}
	\label{fig:T3.6}
\end{figure}

\begin{figure}[!htbc]
  \centering
  \includegraphics[width=.72\textwidth]{{../output/plots_for_paper_von_neumann/part_a/converged_T2.50}.pdf}
	{\par\nobreak\rule[9pt]{35em}{0.5pt}\vspace{-5mm}}
	\caption{Lattice at $T=2.5$. At lower $T$ the lattice becomes increasingly uniform.}
	\label{fig:T2.5}
\end{figure}

\begin{figure}[!htbc]
  \centering
  \includegraphics[width=.72\textwidth]{../output/plots_for_paper_von_neumann/part_a/M_vs_T.pdf}
	{\par\nobreak\rule[9pt]{35em}{0.5pt}\vspace{-5mm}}
	\caption{$M/N$ vs $T$. Notice the phase transition between $\langle s \rangle = 0$ and $\left|\langle s \rangle\right| = 1$ at $T_{C} = 3.50$ K. For this simulation and RNG seed the lattice transitioned to the $\langle s \rangle = 1$ spin up magnetization state.}
	\label{fig:T2.5}
\end{figure}


\clearpage

\subsection{Part B}
\label{subsec:results_part_b}
TODO


\begin{figure}[!htbc]
  \centering
  \includegraphics[width=.72\textwidth]{../output/plots_for_paper_von_neumann/part_b/CT_for_n10.pdf}
	{\par\nobreak\rule[9pt]{35em}{0.5pt}\vspace{-5mm}}
	\caption{$C\left(T\right)$ for $n=10$. This is an example of the noise present at low $n$.}
	\label{fig:CT_n10}
\end{figure}

\begin{figure}[!htbc]
  \centering
  \includegraphics[width=.72\textwidth]{../output/plots_for_paper_von_neumann/part_b/CT_for_n50.pdf}
	{\par\nobreak\rule[9pt]{35em}{0.5pt}\vspace{-5mm}}
	\caption{$C\left(T\right)$ for $n=50$, sharpest peak.}
	\label{fig:CT_n50}
\end{figure}

\begin{figure}[!htbc]
  \centering
  \includegraphics[width=.72\textwidth]{../output/plots_for_paper_von_neumann/part_b/CT_for_n500.pdf}
	{\par\nobreak\rule[9pt]{35em}{0.5pt}\vspace{-5mm}}
	\caption{$C\left(T\right)$ for $n=500$.}
	\label{fig:CT_n500}
\end{figure}


\begin{figure}[!htbc]
  \centering
  \includegraphics[width=.72\textwidth]{../output/plots_for_paper_von_neumann/part_b/Cmax_over_N_vs_n.pdf}
	{\par\nobreak\rule[9pt]{35em}{0.5pt}\vspace{-5mm}}
	\caption{$C_{\mathrm{max}}/N$ vs $\log(n)$. The linear fit is consistent with $C_{\mathrm{max}}/N \sim \log(n)$, for low $n$.}
	\label{fig:Cmax_over_N_vs_n}
\end{figure}



\clearpage
\section{Conclusions}
\label{sec:Conclusions}
TODO

The Python source code used to produce these results can be found online at \url{http://github.com/mepland/PHYS_566_Computational_HW/tree/master/final/code}, and is included in Section~\ref{sec:code}.


\section{Supporting Material}
\label{sec:Supporting_Material}

\lstinputlisting[style=output,label={lst:output}]{../output/plots_for_paper_von_neumann/part_a/part_a.log}
\lstinputlisting[style=output,label={lst:output}]{../output/plots_for_paper_von_neumann/part_b/part_b.log}


\clearpage
\includepdf{../final.pdf}

\clearpage
\section{Code}
\label{sec:code}

% TODO clearpages
\lstinputlisting[style=python]{../code/metropolis_ising.py}
\lstinputlisting[style=python]{../code/sweepmodule.c}
\lstinputlisting[style=python]{../code/module_setup_for_sweepMod.py}

\end{document} %%% end of doc %%%
%%%%%%%%%%%%%%%%%%%%%%%%%%%%%%%%%%%%%%%%%%%%%%%%%%%%%


\bibliographystyle{bib_files/styles/atlasBibStyleWoTitle}
\bibliography{bib_files/my_bib.bib}


\begin{figure}[!htbc]
  \centering
  \includegraphics[width=.72\textwidth]{part_a/uniform_N_1000_bins_10.pdf}
	{\par\nobreak\rule[9pt]{35em}{0.5pt}\vspace{-5mm}}
	\caption{Uniform random number distribution with $N = 1000$ and $10$ bins.}
	\label{fig:uniform_N_1000_bins_10}
\end{figure}


